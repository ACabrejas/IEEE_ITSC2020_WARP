%%%%%%%%%%%%%%%%%%%%%%%%%%%%%%%%%%%%%%%%%%%%%%%%%%%%%%%%%%%%%%%%%%%%%%%%%%%%%%%%
%2345678901234567890123456789012345678901234567890123456789012345678901234567890
%        1         2         3         4         5         6         7         8

%\documentclass[letterpaper, 10 pt, conference]{ieeeconf}  % Comment this line out if you need a4paper

\documentclass[a4paper, 10pt, conference]{ieeeconf}      % Use this line for a4 paper

\IEEEoverridecommandlockouts                              % This command is only needed if 
                                                          % you want to use the \thanks command

\overrideIEEEmargins                                      % Needed to meet printer requirements.

\usepackage{cite}
\usepackage{amsmath,amssymb,amsfonts}
\usepackage{algorithmic}
\usepackage{graphicx}
\usepackage{textcomp}
\usepackage{amsmath}
\usepackage{stfloats}
\usepackage[]{algorithm2e}
\usepackage{subcaption,siunitx,booktabs}
\usepackage{tabularx, booktabs} 
\usepackage{amsfonts}
\usepackage{siunitx}

\newcommand{\bftab}{\fontseries{b}\selectfont}


\def\BibTeX{{\rm B\kern-.05em{\sc i\kern-.025em b}\kern-.08em
    T\kern-.1667em\lower.7ex\hbox{E}\kern-.125emX}}
\graphicspath{{./images/}}
\newcolumntype{C}{>{\centering\arraybackslash}p{1.75cm}}
\usepackage{hyperref}
\hypersetup{pdftex,colorlinks=true,allcolors=blue}
\usepackage{hypcap}
%In case you encounter the following error:
%Error 1010 The PDF file may be corrupt (unable to open PDF file) OR
%Error 1000 An error occurred while parsing a contents stream. Unable to analyze the PDF file.
%This is a known problem with pdfLaTeX conversion filter. The file cannot be opened with acrobat reader
%Please use one of the alternatives below to circumvent this error by uncommenting one or the other
%\pdfobjcompresslevel=0
%\pdfminorversion=4

% See the \addtolength command later in the file to balance the column lengths
% on the last page of the document

% The following packages can be found on http:\\www.ctan.org
%\usepackage{graphics} % for pdf, bitmapped graphics files
%\usepackage{epsfig} % for postscript graphics files
%\usepackage{mathptmx} % assumes new font selection scheme installed
%\usepackage{times} % assumes new font selection scheme installed
%\usepackage{amsmath} % assumes amsmath package installed
%\usepackage{amssymb}  % assumes amsmath package installed

\title{\LARGE \bf
WARP method for separating recurrent and outstanding congestion in Motorways and long-term travel time estimation*
}


\author{Alvaro Cabrejas Egea$^{1}$ and Colm Connaughton$^{2}$% <-this % stops a space
\thanks{*This work was part funded by the EPSRC under grant no. EP/L015374.}% <-this % stops a space
\thanks{$^{1}$Alvaro Cabrejas Egea is with MathSys Centre for Doctoral Training, University of Warwick, CV4 7AL, Coventry, United Kingdom and The Alan Turing Institute, NW1 2DB, London, United Kingdom
        {\tt\small a.cabrejas-egea@warwick.ac.uk}}%
\thanks{$^{2}$ Colm Connaughton with the Warwick Mathematics Institute \& Centre for Complexity Science, University of Warwick,
        CV4 7AL, Coventry, United Kingdom
        {\tt\small c.p.connaughton@warwick.ac.uk}}%        
}


\begin{document}



\maketitle
\thispagestyle{empty}
\pagestyle{empty}


\begin{abstract}
Accurate highway travel times estimation is a requirement for a working for Advanced Traveler Information Systems (ATIS).
Here we present the WARP method for separating recurrent and outstanding congestion from a travel time time series using wavelet analysis, and then produce estimates for the following week in an accurate and computationally inexpensive manner. 
We train and test this method using 12 weeks of real time travel data gathered at link level in the M6 and M11 in the United Kingdom.
Our approach makes use of the travel time series gathered and published by the UK's National Traffic Information Service (NTIS) via time series analysis.
In these series we can usually find a noisy background demonstrating seasonal pattern with occasional large deviations generated by motorway congestion.
The separation algorithm here presented uses statistical analysis of the Wavelet Transform of the travel times signal across different time scales to split the original signal into two signals in the time-frequency domain: background and spikes, and then inverse transform them to use them for estimation.
We process and combine 8 weeks of these signals, using spectral filtering and non-parametric locally weighted regression (LWR) to predict travel time in the link in the following week with a resolution of one minute, repeating this process 4 times on a rolling basis.
We find that this method significantly reduces the estimation error when compared to those produced using methods based on time segmentation and those estimates made public by the NTIS system.
\end{abstract}
\section{Introduction} \label{Introduction}
\subsection{Background} \label{Background}
Highways England (HE) is responsible for most Motorways and major roads in England, comprising the Strategic Road Network (SRN).
The SRN is monitored by the National Traffic Information System \cite{NTIS}, which gathers and processes travel time, speed, flow and overhead space data in real time using road sensors and special vehicles.
The smallest components of the SRN are the "links", segments of motorway ranging from 500 to 20000 metres long.
NTIS uses the historic series of real time logged data to assign each link in the network with a traffic profile.
Traffic profiles contain, for each minute of a day in a given date, the expected travel time to transverse the link to which it is associated. 
These profile values are published together with the raw gathered average travel times, but the methodology followed to calculate one from the other is not publicly available.

While the profiles, in fairly used roads, have a great variation depending on the time of the day and the day of the week, their inter-week variation remains low, with major trends changes taking longer (order of months) than the biggest temporal unit considered (weeks). 
This display a slow rate of change with respect to the data frequency, meaning that to calculate a profile is significantly different than short-term forecasting.
Since there is no a-priori mathematical definition of traffic profiles (common travel times), here, we will define it as the collection of values that jointly minimise the Root Mean Squared Error (RMSE) or Mean Absolute Relative Error (MARE), when comparing it with the subsequently sampled real travel times.

In this paper, we follow up on the future work in \cite{ttprofiles}, providing a novel algorithm capable of generating one week of expected travel times, using a single parameter and using only governmental agencies publicly available data.
This approach stems from spectral and statistical analysis of recent historical data and removes the requirement of any prior form of segmentation of times and days into different classes. 
Instead, the algorithm relies on pattern discovery for intra- and inter-day variability which it computes directly from input data.
\subsection{Previous Work} \label{Previous Work}
There is extensively available literature on travel times, albeit recent research is mostly focused on forecasting in the shorter term, with lesser presence of long term estimation studies \cite{long-term} \cite{long-term-2}. 
Machine learning and statistical analysis methods receive the most attention \cite{should}. In machine learning, neural networks are recently having a high relevance \cite{NN} \cite{spectral2}.
Other approaches are closer to the methods here presented: making use of historical data \cite{simple} \cite{dynamic-historic}, differentiating between rush and non-rush hour \cite{peak-historic}, using spectral methods \cite{spectral1} or Locally Weighted Regressions \cite{williams} \cite{sun} \cite{zhong} \cite{chowdhury} \cite{acqua} \cite{vana}.
Wavelet use has been previously found useful in combination with Kalman filters \cite{nonlinear}, neural networks \cite{samant} \cite{ghosh} \cite{hojjat} \cite{adeli} and statistical analysis \cite{basu} \cite{hang}; but these either cover short predictions, use the Wavelet analysis of travel times for other purposes such as incident detection, or don't focus on the spectral properties across timescales or are not real-world data based.
Relevant comparisons involving some of these studies are performed in \cite{nikovski}, \cite{lint}, \cite{mori} and \cite{ser}. 
Among these methods, the transferability to other locations is not often evaluated, being tuned for a specific location with its own specific conditions.
In order to ensure transferability, our method is examined on 39 individual locations across two motorways and independently scored for each site.
The work presented here, makes use of a combination of Continuous Wavelet Transform \cite{morletwavelet}, tree decisions, spectral analysis, and Locally Weighted Regression (LWR).
\section{Travel times in Motorways} \label{Travel times in Motorways}
Vehicles' travel times provide us with the most straightforward way of measuring the state of the traffic flow in a length of road.
The average travel time in a given minute of the day on a specific section of road, will be the average time to transverse the space from the entry to the exit loops for all vehicles that passed through.
\begin{figure}[htbp]
\centerline{\includegraphics[width=\linewidth]{./images/Travel_Time.pdf}}
\caption{Link 1170079, M6. Travel Time over 28 days of minutely data between 07/Mar/2016 and 04/Apr/2016.}
\label{fig:travel_time}
\end{figure}
In Figure \ref{fig:travel_time}, it can be observed that most of the time, the travel time in a link follows vaguely a repeating pattern during most days, with minima located at night matching the bounded free flow time (except for speeding drivers).
As the morning rush starts, travel time will rise leading to the morning traffic jams which are generated from the collective drivers behaviour.
More effects of this collective dynamics can be observed in the afternoon during the evening rush; normally being able to find a plateau between these two peaks.
Finally, travel times progressively decay towards the night's free-flow regime.
In Figure \ref{fig:travel_time}, we observe a series of peaks found outside of this normally bounded yet oscillating behaviour. 
Travel time in these events can climb up to several times fold the normal amount.
The predictability in terms of inter-oscillation period and amplitude of these larger oscillations is much lower than the recurrent component described above.
\begin{figure}[htbp]
	\centerline{\includegraphics[width=\linewidth]{./images/ACF_M6_Link3.pdf}}
	\caption{Autocorrelation function of Travel Time in figure \ref{fig:travel_time} with a maximum lag of 4 weeks (1440 points/day). Double seasonality can be observed in the weekly and daily periods.}
	\label{fig:ACF}
\end{figure}
Lastly, in figure \ref{fig:ACF}, we can see the Autocorrelation Function for the Travel Time, plotted with a maximum lag of 4 weeks. Here we can see a double seasonal pattern with periods of 1 day and 1 week.
This regularity seen in time travel time series can and is often used by modellers to approximate and forecast travel times as is explored over the next sections.
\section{Basic methods for profile estimation}\label{Basic methods for profile estimation}
\subsection{Exponentially Weighted Moving Average for Profiles} \label{ewma}
As explained in \cite{ttprofiles}, a basic approach to estimating profiles is to apply an Exponentially Weighted Moving Average (EWMA) across a given minute of the available days, assuming that similar behaviour is to be expected at similar times on different days. 
Then, the profile estimation $\hat{x}(i,d+1)$ for the $i-th$ minute of a date $d$, controlling our memory parameter $\alpha \in [0,1]$ to balance the memory of the process and with measured travel time $x_i^d$, will be:
\begin{equation}
\begin{aligned}
\hat{x}^{d+1}_i \!&= \alpha  x^{d}_{i} + (1-\alpha)\hat{x}^{d}_{i} \\ 
&= \alpha  x^{d}_{i} + \alpha^2  x^{d-1}_{i} + (1-\alpha - \alpha^2) \hat{x}^{d-1}_{i}\\ 
&= ...
\end{aligned}
\label{eq:ewma}
\end{equation}
EWMA-based profiles have a main issue: disruptions after large events generated due to the way in which the memory decays as we can see above in equation \ref{eq:ewma}.
\begin{figure}[htbp]
	\centerline{\includegraphics[width=\linewidth]{./images/EWMA.pdf}}
	\caption{False EWMA predictions after a large deviation.}
	\label{fig:EWMA}
\end{figure}
Recent measurements receive exponentially greater weights than events farther in the past. 
If an large deviation from the baseline patter occurs, subsequent estimations will have a bias towards partially replicating the deviation and consistently predicting over-estimates until newer data is included and the effect dissipates.
\subsection{Time-based Segmentation}\label{segmentation}
Furthermore, to recognize and use the precise characteristics of specially distinct dates, these methods make heavy use of segmentation based on the date. 
Days are grouped according to categories defined by traffic authorities and the EWMA is applied across dates following in the same category (i.e. Saturdays, weekdays, New Year, ...).
This segmentation in classes, joint with the weakness commented in Section \ref{ewma} can generate lasting perturbations, propagating across weeks into future estimations, yet never being reflected on the travel times measurements. A reconstruction of this event is shown in Figure \ref{fig:EWMA}.
Consequently, significant operational and geographical expertise about specific roads is needed to create a valid segmentation, given that the EWMA approach is effective exclusively for recurrent congestion. 
These requirements often leads to the generation of legacy systems which become increasingly difficult to maintain as time passes, their usefulness declining over time or requiring additional efforts for continually train new staff to maintain the system.
These modelling and operational limitations make Segmentation and EWMA based profiles suboptimal both in performance and operation.
\section{Data Selection and Contents}
Motorways M6 and M11 in England were chosen due to their high use and their display of both recurrent and outstanding congestion, being key in several heavily used commuting routes. 
\begin{itemize}
	\item The dataset aggregates 90 days (12 complete weeks) minutely entries (07/03/2016-05/06/2016).
	\item Links with over 10\% of data missing or containing access ramps were discarded.
	\item The previous condition left 14 different links in the M6 and 25 links in the case of the M11.
	\item Entries missing for 10 minutes or less were linearly interpolated.
	\item Entries missing for over 10 minutes were left as missing values.
\end{itemize}
The algorithm uses 8 weeks of data to predict one entire week ahead.
After a week, the oldest week is deleted, and the most recent one is incorporated, producing a new estimate for the subsequent week.
This procedure is simulated 4 times.
z
For each link-date pair, the data comprises minutely data, containing:
\begin{itemize}
	\item Average vehicle travel time in seconds
	\item Profile (expected) travel time in seconds
	\item Traffic Flow in cars/hour
	\item Overhead Space in meters
\end{itemize}

\section{Background and Spikes}
\subsection{Characteristics}
As described in \cite{ttprofiles}, if we look at the travel times, we find they operate in two different regimes that we denominated Background and Spikes.
The background is characterised by a stability around a mean value with small oscillations in amplitude but great in frequency, making it suitable for seasonal analysis and spectral filtering (smoothing).
The spikes on the other hand, are zero most of the time but can quickly climb to extreme values. Their oscillations show much greater amplitude and much lower frequency, creating long reaching effects. They are non periodic as per detection in the time-frequency domain, however, non harmonic seasonality can be extracted via non-parametric regression.

In this context, if we assume Gaussian noise $\xi$, and given the additive properties of wavelet decomposition, the decomposition will be of the form:
\begin{equation}
\textrm{Travel Time}_t  = \textrm{Background}_t + \textrm{Spikes}_t + \xi
\end{equation}
The objective is to separate the signals such that the times of smooth non-congested operation, together with the recurring congestion are captured in the Background and passed through spectral smoothing, mitigating the estimation errors created by the high frequency oscillations and achieving a view of what can be daily observed, so seasonal patterns can be extracted in the shorter and longer periods shown in Fig. \ref{fig:ACF}.
Meanwhile, the spikes, containing the non-recurring congestion, can be searched for any seasonality left time scales larger than the period in which the travel times oscillate, as also suggested by Fig. \ref{fig:ACF}. 
If performed correctly, the remainder after this seasonal extraction step of the decomposition, should contain only isolated events with large deviations from the profile and white noise.
\subsection{Wavelet Time Series Decomposition}\label{decomposition}
\begin{figure*}[htbp]
	\centerline{\includegraphics[width=\linewidth]{./images/WT_combined_1500_600.png}}
	\caption{Left: Power of DWT of the original Travel Time series for Link 1170079 of the M6.
	Right: Power of DWT of the extracted Background $\vec{B_t}$ series, as detailed in Section \ref{decomposition}, for Link 1170079 of the M6}
	\label{fig:wt}
\end{figure*}
To perform the decomposition, we will take advantage of the additive properties of the wavelet transform. First the time series $\vec{x_t}$ of length $t$, and elements $x_i$ with $i \in [0,t]$ is turned into a zero mean series $\vec{x_t} - mean(\vec{x_t})$ and used as input for a Continuous Wavelet Transform (CWT) \cite{daubechies} \cite{mallat} using a Morse wavelet \cite{morse} and 140 timescales levels.
The output of this first transform is a complex matrix $\vec{W}_{\textrm{levels}\,\times\, t}$, for the elements of which we calculate their modulus $\rho$, phase $\phi$ and power $P$. 
A heatmap of $P^2$ for the original time series can be seen in the first subplot in Figure \ref{fig:wt}.
In the figure we can observe that the most influential dynamics occurring during the series length (x axis), happen at the timescales (y axis) where it was expected based on Figure \ref{fig:ACF}, namely 1 day (1440 minutes) and 1 week (10080 minutes).
In the figure, we also observe that surges in power across different timescales occur at the same time as the non-recurring congestion, since in order to approximate this signal, the CWT algorithm needs to combine several wavelets with smaller periods than those dominating the recurring part of the series.

In order to isolate this non-recurring component, the series is sequentially assessed over all the time domain by taking a horizontal slice for a single timescale level $l$, and generating a series $\vec{x_{t}}^l$.
After fixing $l$, we calculate the Median and Inter Quantile Range of $\vec{x_{t}}^l$ to search for outliers.
A maximum threshold value is set equal to $T = median(\vec{x_t}) + \alpha * IQR(\vec{x_{t}}^l)$, with $\alpha \in [0,\inf]$ being a parameter that defines how aggressively we target spikes. 
Then, the individual elements of $\vec{x_{t}}^l$, $x_{i}^l$ are individually evaluated: if found below the limit, they are stored in a Background container $B_i^l = x_{i}^l$; otherwise, the fraction below the threshold will be nonetheless passed on to the background $\vec{B_i}^l = median(\vec{x_{t}}^l) + \alpha * IQR(\vec{x_{t}}^l)$, with the remaining part going into the Spikes storage $\vec{S_i}^l = x_{i}^l - B_{i}^l$.

In the extremes, for $\alpha = 0$ we would find that any deviation from the median is taken into the spikes signal, and with $\alpha = \infty$ only points infinitely distant from the mean would be taken into the spikes.
The results produced here were obtained with $\alpha=1$ to showcase the most basic setup and the effectiveness of even an non-tuned version.

Once all levels have been processed in this manner, we use the previous information about $\phi$ to reconvert the two series from being characterised in terms of $(\rho, \phi)$ to the complex components of the DWT.
After this step, we apply the Inverse DWT to $\vec{B_t}$ and $\vec{S_t}$, obtaining the two series that can be observed in Figure \ref{fig:splitting}.
\subsection{Analysis of Extracted Background and fitness of method}
The results of the separation of the background can be seen in Fig. \ref{fig:wt}. Here it can be observed that the separated background is still characterised by seasonality dominating the weekly (10080 minutes) and daily (1440 minutes) timescales, and keeping a very similar structure in the regions occupied by the faster dynamics ($<$ 120 minutes) to that of the original series.
Simultaneously, the surges in Power across timescales of the original DWT-transformed series (left), which are associated with congestion events, have been greatly reduced or eliminated altogether in the second subplot.

In the author's view, this demonstrates the applicability of the method to process a timeseries showing multiple seasonalities and punctuated by large deviations from the normal dynamics of the process into two separate series, one of which will exclusively contain the baseline dynamics of the process, respecting its structure and natural variability, and the other will only contain those large events that occur outside of the smooth operational region.
\subsection{Series Recombination}
\begin{figure}[htbp]
	\centerline{\includegraphics[width=\linewidth]{./images/Splitting.png}}
	\caption{Recombination post Inverse Wavelet Transform}
	\label{fig:splitting}
\end{figure}
Due to small noise introduction during the inverse transform a threshold is applied to $\vec{S_t}$, where elements representing spikes of less than 3 seconds in amplitude are set to zero when looking at future estimation.

For future prediction steps, an indicator function is defined for every entry in the series, taking value:
\begin{equation}
    \delta_i^{spike}=
    \begin{cases}
      1, & \text{if } x_i > \text{Threshold}\\
      0, & \text{otherwise.}
    \end{cases}
    \label{delta}
  \end{equation}
\section{WARP Travel Time Prediction Algorithm} \label{algorithm}
The objective now is, given the seasonality and separation of the original signal provided above, to generate a time travel prediction model that accounts for the cyclic variations and recurrent congestion but remains resilient to unexpected deviations and rare events.

We aim to provide a robust algorithm that mitigates the propagation of extreme events into the future (unlike EWMA). 
It must work for all locations and not require the use of time segmentation. The trend term must be nearly flat, based on the different timescales for the growth of demand on a Motorway level and the seasonalities concerning this paper. 
Finally, it must have uncorrelated residuals, Gaussian distributed with mean 0.
Following these requirements, we introduce the WARP algorithm (Wavelet Augmented Regression Profiling) from section \ref{algostart}.
\subsection{Naive Segmentation (Null)}
For an accurate performance comparison of the WARP algorithm, we implemented a basic segmentation based algorithm. 
It involves applying uniform weights to the training data from previous weeks. 
On the $i-th$ minute of a week and using the previous $n$ weeks (8 in this paper) as training, the Naive Segmentation (NS) profile is:
\begin{equation}
\hat{x}(i,n) = \sum_{\textrm{week}=1}^{n} \frac{x^i_n}{n} 
\end{equation}

\subsection{WARP: Spectral Component}
\label{algostart}
The Background signal shows oscillations of high frequency and low amplitude almost ubiquitously. 
It can be smoothed by discarding, in the frequency domain, the frequencies that in which the oscillations occur and those outside the scope of this study (over 4 weeks and under 4 hours), while keeping those in the information bearing bands by using the Fast Fourier Transform (FFT) \cite{FFT}.
Once this step is performed, and since large events have been removed previously, an EWMA can be applied to the modified weekly Power Spectra in the FFT, to then compute the Inverse FFT Transform to obtain our Background Prediction.
\subsection{WARP: Seasonal Component}
The seasonal component is computed via Seasonal-Trend Decomposition based on LOESS (STL) \cite{STL}.
STL is resilient to outliers, can manage any combination of seasonalities, allowing to control their change over time as well as the smoothness of the trend \cite{forecasting}.

We begin by isolating the daily seasonality $S_d$ from the entire Background training series using STL.
Trend and remainder are summed and re-analysed for weekly seasonality $S_w$, this step also produces a trend series $T_r$ and a remainder which should be Gaussian distributed, zero mean.
Global seasonality calculated as $S_g = S_d + S_w$.
We then average the seasonality over the training weeks to obtain a value for each minute of the week we are to estimate.
Then, $T_r$, nearly flat as per \ref{algorithm}, is linearised to obtain a baseline series $B_t$, and the Baseline prediction is $B_p = B_t + G_s$.
Finally, the Spikes are searched for any seasonality left on the weekly level $Sp_w$, discarding the trend and remainder terms, and obtaining the final seasonal component as $\textit{SEASONAL} = B_t + {Sp}_w$.

\subsection{WARP Hybrid Profile}
\begin{figure}[htbp]
	\centerline{\includegraphics[width=\linewidth]{./images/flow_crop.pdf}}
	\caption{Flowchart of the data streams in the algorithm}
	\label{fig:dataflow}
\end{figure}
The final WARP profile, containing the Spectral and Seasonal components, depends on the regime as per Eq. \ref{delta}:
\begin{equation}
\textrm{WARP} = \textrm{Seasonal} * \delta_{\textrm{spike}} + \textrm{Spectral} * (1 - \delta_{\textrm{spike}})
\end{equation}
\section{Accuracy Results}
\begin{table*}[bp]
	\caption{MARE Distribution in M6 prediction}
	\centering
	\begin{center}
	\begin{tabular}{SSSSSSSS} \toprule
    {$\textit{Profile / MARE}$} & {$(>-25\%]$} & {$(-25\%,-15\%] $} & {$(-15\%,-5\%]$} & {$(-5\%,5\%)$} & {$[5\%,15\%)$} & {$[15\%,25\%)$} & {$[>25\%)$} \\ \midrule
    \textit{Published}  & 1.58 & 0.54 & 5.69 & 56.21 & 30.79 & 3.17 & 2.02 \\
    \textit{Wavelet}  & 1.57  & 0.51 & 3.22  & 81.73 & 12.14 & 0.70  & 0.13 \\ \bottomrule
\end{tabular}
	\end{center}
\end{table*}
\begin{table*}[bp]
	\caption{MARE Distribution in M11 prediction}
	\centering
	\begin{center}
	\begin{tabular}{SSSSSSSS} \toprule
    {$\textit{Profile / MARE}$} & {$(>-25\%]$} & {$(-25\%,-15\%] $} & {$(-15\%,-5\%]$} & {$(-5\%,5\%)$} & {$[5\%,15\%)$} & {$[15\%,25\%)$} & {$[>25\%)$} \\ \midrule
    \textit{Published}  & 0.85 & 1.15 & 19.21 & 62.83 & 13.21 & 1.65 & 1.10 \\
    \textit{Wavelet}  & 0.78  & 0.32 & 3.29  & 81.33 & 12.47 & 1.09  & 0.73 \\ \bottomrule
\end{tabular}
	\end{center}
\end{table*}
Here, we assess the accuracy of the WARP profiles against the NS Model and the Published Profiles.
For every minute of the prediction series on each link, we calculate the Mean Absolute Relative Error (MARE).
The Root Mean Square Error (RMSE) has been calculated on a Motorway level as the average of the RMSE across its links.
\begin{figure}[htbp]
	\centerline{\includegraphics[width=\linewidth]{./images/M6_daytime_8_12.pdf}}
	\caption{MARE over the times of the day for the M6}
	\label{fig:m6dt}
\end{figure}
\begin{figure}[htbp]
	\centerline{\includegraphics[width=\linewidth]{./images/M11_daytime_8_12.pdf}}
	\caption{MARE over the times of the day for the M11}
	\label{fig:m11dt}
\end{figure}
\begin{figure}[htbp]
	\centerline{\includegraphics[width=\linewidth]{./images/M6_quantile_1_ab_8_12.pdf}}
	\caption{MARE across percentiles of travel time for the M6.}
	\label{fig:m6q}
\end{figure}
\begin{figure}[htbp]
	\centerline{\includegraphics[width=\linewidth]{./images/M11_quantile_1_ab_8_12.pdf}}
	\caption{MARE across percentiles of travel time for the M11.}
	\label{fig:m11q}
\end{figure}

In Figures \ref{fig:m6dt} and \ref{fig:m11dt}, WARP displays lower predictive error than the Published Profiles and the NS Model at all times of day and motorways. 
This is most relevant during morning and evening rush hours, where the other profiles predictive error soars, WARP suffers no meaningful performance worsening relative to the plateau in the middle of the day.
The error at rush hours is reduced by a minimum of 50\% across all cases, reaching as high as 63.4\% in the case of the M6 morning rush.

In Figs. \ref{fig:m6q} and \ref{fig:m11q} it can be observed that the predictive accuracy of the WARP profile is significantly higher than thhat of the Published Profiles or the NS Model across nearly all percentiles of travel time. All three suffer similar errors under the most extreme deviations.
This is most notorious in the percentile range $[50-95]$ where the Published Profile performs poorly.
\begin{table}[htbp]
	\caption{MARE per Link on M6 and M11}
	\begin{center}
		\begin{tabular}{|c|c|}
			\hline
			\textbf{Links M6}&{\textbf{MARE}} \\
			\hline
			117007401& 0.0290\\
			\hline
			117007501& 0.0680\\
			\hline
			117007601& 0.0293\\
			\hline
			117007801& 0.0826\\
			\hline
			117007901& 0.0435\\
			\hline
			117008401& 0.0484\\
			\hline
			117009102& 0.0338\\
			\hline
			117011901& 0.0295\\
			\hline
			117012001& 0.0584\\
			\hline
			117012101& 0.0370\\
			\hline
			117012201& 0.0605\\
			\hline
			117012301& 0.0379\\
			\hline
			117016001& 0.0496\\
			\hline
			123025901& 0.0427\\
			\hline
		\end{tabular}
		\quad
		\begin{tabular}{|c|c|}
			\hline
			\textbf{Links M11}&{\textbf{MARE}} \\
			\hline
			199048301& 0.0510\\
			\hline
			199048701& 0.0279\\
			\hline
			199048801& 0.1053\\
			\hline
			199048901& 0.0239\\
			\hline
			199049002& 0.0306\\
			\hline
			199049101& 0.0213\\
			\hline
			199049402& 0.0223\\
			\hline
			199049501& 0.0181\\
			\hline
			199049702& 0.0297\\
			\hline
			199049801& 0.0408\\
			\hline
			199050002& 0.0445\\
			\hline
			199050101& 0.0272\\
			\hline
			199050202& 0.0288\\
			\hline
			199050901& 0.0433\\
			\hline
			199063301& 0.1203\\
			\hline
			199063701& 0.0500\\
			\hline
			199063801& 0.0265\\
			\hline
			199064203& 0.0230\\
			\hline
			199065202& 0.0259\\
			\hline
			200021668& 0.0280\\
			\hline
			200024801& 0.0233\\
			\hline
			200028639& 0.0241\\
			\hline
			200028641& 0.0188\\
			\hline
			200028645& 0.0435\\
			\hline
			200028648& 0.0499\\
			\hline
		\end{tabular}
		\label{tab1}
	\end{center}
	\label{table:m6mape}
\end{table}
\begin{table}[htbp]
	\caption{Global MARE \& RMSE per Motorway}
	\begin{center}
		\begin{tabular}{|c|c|c|}
			\hline
			\textbf{Motorway}&{\textbf{MARE}}&{\textbf{RMSE [s]}} \\
			\hline
			M6& 0.0385& 0.0464\\
			\hline
			M11& 0.0379& 0.0484\\
			\hline
		\end{tabular}
		\label{mapeglobal}
	\end{center}
\end{table}
\section{Conclusion and Future Work}
The algorithm presented above meets the requirements described in Section \ref{algorithm} and earlier defined in \cite{ttprofiles}, albeit still keeping one parameter.
In the future, sensitivity analysis could explore the limits of the algorithm in terms of minimum training data set, as well as maximum performance with increased training.
Also, more sophisticated analysis of the Wavelet Transform series can be achieved, focusing on the information-bearing bands and using adaptive thresholds for the separation.
Given the flexibility offered by this method, it could be extended for incident detection or adapted to other problems besides road traffic.
\clearpage
\begin{thebibliography}{00}
\bibitem{NTIS} The Highways Agency, "National Transport Information System Publish Services", Technical Report, 2011. 
\bibitem{ttprofiles} Cabrejas-Egea, A. , de Ford, P., \& Connaughton, C. (2018, November). Estimating Baseline Travel Times for the UK Strategic Road Network. In 2018 21st International Conference on Intelligent Transportation Systems (ITSC) (pp. 531-536). IEEE.
\bibitem{long-term} J. Moreira, A. Jorge, J. Sousa, C. Soares 2012. Comparing state-of-the-art regression methods for long term travel time prediction. Intelligent Data Analysis. 16. 10.3233/IDA-2012-0532. 
\bibitem{long-term-2} G. Klunder, P. Baas, F. op de Beek. 2018. A long-term travel time prediction algorithm using historical data. 14th World Congress on Intelligent Transport Systems, ITS 2007; Beijing; China; October 2007 through 13 October 2007, 1191-1198
\bibitem{should} Kirby, H. R., Watson, S. M., Dougherty, M. S. 1997. Should we use neural networks or statistical models for short-term motorway traffic forecasting?. International Journal of Forecasting, 13(1), 43-50.
\bibitem{NN} Lee, Y. 2009. Freeway travel time forecast using artifical neural networks with cluster method. In Information Fusion, 2009. FUSION'09. 12th International Conference on (pp. 1331-1338). IEEE.
\bibitem{spectral2} Park, D. et al. 1999. Spectral basis neural networks for real-time travel time forecasting. Journal of Transportation Engineering, 125(6).
\bibitem{simple} Rice, J., Van Zwet, E. 2004. A simple and effective method for predicting travel times on freeways. IEEE Transactions on Intelligent Transportation Systems, 5(3), 200-207.
\bibitem{dynamic-historic} Chien, S. I. J. et al. 2003. Dynamic travel time prediction with real-time and historic data. Journal of transportation engineering, 129(6), 608-616.
\bibitem{peak-historic} Zhang, Y. et al. 2011. Analysis of peak and non?peak traffic forecasts using combined models. Journal of Advanced Transportation, 45(1).
\bibitem{spectral1} Nicholson, H., Swann, C. D. 1974. The prediction of traffic flow volumes based on spectral analysis. Transportation Research, 8(6), 533-538.
\bibitem{williams} Williams, B., Durvasula, P., Brown, D. 1998. Urban freeway traffic flow prediction: application of seasonal autoregressive integrated moving average and exponential smoothing models. Transportation Research Record: Journal of the Transportation Research Board, (1644), 132-141.
\bibitem{sun} Sun, H., Liu, H. X., Xiao, H., He, R. R., Ran, B. 2003. Short term traffic forecasting using the local linear regression model. In 82nd Annual Meeting of the Transportation Research Board, Washington, DC.
\bibitem{zhong} Zhong, M., Sharma, S., Lingras, P. 2005. Refining genetically designed models for improved traffic prediction on rural roads. Transportation planning and technology, 28(3), 213-236.
\bibitem{chowdhury} Chowdhury, N. K., Nath, R. P. D., Lee, H., Chang, J. 2009. Development of an effective travel time prediction method using modified moving average approach. In International Conference on Knowledge-Based and Intelligent Information and Engineering Systems (pp. 130-138). Springer, Berlin, Heidelberg.
\bibitem{acqua} Dell'Acqua, P., Bellotti, F., Berta, R., De Gloria, A. 2015. Time-aware multivariate nearest neighbor regression methods for traffic flow prediction. IEEE Transactions on Intelligent Transportation Systems, 16(6), 3393-3402.
\bibitem{vana} Kumar, S. V., Vanajakshi, L. 2015. Short-term traffic flow prediction using seasonal ARIMA model with limited input data. European Transport Research Review, 7(3), 21.
\bibitem{nonlinear} Li, Sheng. "Nonlinear combination of travel-time prediction model based on wavelet network." Proceedings. The IEEE 5th International Conference on Intelligent Transportation Systems. IEEE, 2002.
\bibitem{samant} Samant, A., and H. Adeli. "Feature extraction for traffic incident detection using wavelet transform and linear discriminant analysis." Computer‐Aided Civil and Infrastructure Engineering 15.4 (2000): 241-250.
\bibitem{ghosh} Ghosh‐Dastidar, Samanwoy, and Hojjat Adeli. "Wavelet‐clustering‐neural network model for freeway incident detection." Computer‐Aided Civil and Infrastructure Engineering 18.5 (2003): 325-338.
\bibitem{hojjat} Jiang, Xiaomo, and Hojjat Adeli. "Dynamic wavelet neural network model for traffic flow forecasting." Journal of transportation engineering 131.10 (2005): 771-779.
\bibitem{basu} Tchrakian, Tigran T., Biswajit Basu, and Margaret O'Mahony. "Real-time traffic flow forecasting using spectral analysis." IEEE Transactions on Intelligent Transportation Systems 13.2 (2012): 519-526.
\bibitem{hang} Yang, Hang, et al. "A hybrid method for short-term freeway travel time prediction based on wavelet neural network and Markov chain." Canadian Journal of Civil Engineering 45.2 (2017): 77-86.
\bibitem{adeli} Dharia, Abhijit, and Hojjat Adeli. "Neural network model for rapid forecasting of freeway link travel time." Engineering Applications of Artificial Intelligence 16.7-8 (2003): 607-613.
\bibitem{nikovski} Nikovski D, Nishiuma N, Goto Y, Kumazawa H. Univariate short-term prediction of road travel times. Intelligent Transportation Systems, 2005. Proceedings. 2005 Sep 13 (pp. 1074-1079). IEEE.
\bibitem{lint} van Hinsbergen, C.P.IJ., van Lint, J.W.C.,  Sanders, F. M. Short Term Traffic Prediction Models. ITS World Congress, Beijing, China. 2007.
\bibitem{mori} Mori U, Mendiburu A, Alvarez M, Lozano JA. A review of travel time estimation and forecasting for Advanced Traveller Information Systems. Transportmetrica A: Transport Science. 2015 Feb 7;11(2):119-57.
\bibitem{ser} Lana I, Del Ser J, Velez M, Vlahogianni EI. Road Traffic Forecasting: Recent Advances and New Challenges. IEEE Intelligent Transportation Systems Magazine. 2018;10(2):93-109.
\bibitem{morletwavelet} Grossmann, Alexander, and Jean Morlet. "Decomposition of Hardy functions into square integrable wavelets of constant shape." SIAM journal on mathematical analysis 15.4 (1984): 723-736.
\bibitem{daubechies} Daubechies, Ingrid. Ten lectures on wavelets. Vol. 61. Siam, 1992.
\bibitem{mallat} Mallat, Stéphane. A wavelet tour of signal processing. Elsevier, 1999.
\bibitem{morse} Olhede, Sofia C., and Andrew T. Walden. "Generalized morse wavelets." IEEE Transactions on Signal Processing 50.11 (2002): 2661-2670.
\bibitem{STL} R. B. Cleveland, W. S. Cleveland, J. E. McRae and I. Terpenning, "STL: A Seasonal-Trend Decomposition Procedure Based on Loess", Journal of Official Statistics, vol. 6, 1990, pp. 3-73.
\bibitem{forecasting} R. J. Hyndman and G. Athanasopoulos, "Forecasting: Principles and Practice", Otexts, 2013.
\bibitem{FFT} J. W. Cooley and J. W. Tukey, "An algorithm for the machine calculation of complex Fourier series", Mathematics of computation, 1965, vol. 19, no 90, p. 297-301.
\end{thebibliography}
\end{document}
